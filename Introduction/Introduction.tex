The development of autonomous driving has experienced significant progress in the last decade, propelled by breakthroughs in artificial intelligence, machine learning, and sensor technology. Advancements in the pursuit of creating completely self-driving vehicles that can safely maneuver through intricate surroundings have resulted in notable achievements. Nevertheless, despite these technological breakthroughs, there are still some significant obstacles that hinder the mainstream acceptance and implementation of autonomous driving systems. This literature analysis emphasizes the significance of tackling these obstacles in order to make significant progress towards achieving fully autonomous cars.

We prioritize the CARLA simulator due to its practicality in autonomous driving research. CARLA is a open-source urban driving simulator that creates a wide variety of driving situations, ranging from ordinary to extremely intricate, which are essential for testing and verifying autonomous driving algorithms. The high-fidelity simulation environment enables researchers to conduct experiments with different conditions and edge situations that are challenging to encounter in real-world testing.
CARLA's evaluation framework has three fundamental metrics: Driving Score, Route Completion, and Infraction Penalty. Driving Score assesses overall driving performance by combining route completion and infraction penalties. Route Completion measures the percentage of the predefined route that the autonomous vehicle successfully completes. Infraction Penalty quantifies the number and severity of traffic infractions committed by the autonomous vehicle.

By employing these measures, researchers may thoroughly assess the efficiency and security of self-driving systems in a regulated, simulated setting, facilitating progress in practical, real-life implementations. Therefore, we have chosen the models that exhibit the highest performance in the CARLA simulator. We survey how these models mitigate the following challenges: handling rare events, sensor fusion, addressing model training biases, and limitations of adopting control-based or trajectory-based approaches alone.