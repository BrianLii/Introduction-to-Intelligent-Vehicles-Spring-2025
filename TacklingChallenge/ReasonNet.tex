The majority of current techniques largely concentrate on addressing common and regular driving scenarios, while inadequately handling infrequent or unexpected events, such as quick pedestrian movements from obstructed locations or unanticipated maneuvers by other vehicles. The lack of this capability highlights the urgent requirement for a more comprehensive and unified method to improve the accuracy of perception and prediction in real-time, especially in crowded and intricate urban traffic situations.

ReasonNet \cite{shao2023reasonnet}, an innovative driving framework, tackles these difficulties by utilizing both temporal and global scene information. The objective is to offer accurate forecasts of how the environment will change and enhance the resilience of decision-making in different urban driving scenarios.

\subsubsection{Brief Explanation of ReasonNet}

\paragraph{Input and Output Representations.} ReasonNet primarily relies on sensor data from a LiDAR sensor and many RGB cameras as its main inputs. The vehicle's left, front, right, and rear cameras capture images from distinct viewpoints, while the LiDAR sensor generates point cloud data for the creation of 3D models of the environment. ReasonNet generates multiple types of outputs to facilitate safe and efficient driving. These include a path prediction with multiple waypoints for the vehicle to follow, an occupancy map indicating the presence of objects in the vicinity, and traffic rule information such as the status of traffic lights and stop signs. The path prediction guides the vehicle's navigation, while the occupancy map and traffic rule information help in making real-time driving decisions.

\paragraph{Model Architecture.} ReasonNet is built upon three fundamental modules: The Perception Module, Temporal Reasoning Module, and Global Reasoning Module.

The Perception Module processes and combines sensor information to build BEV features. The image data use a two-dimensional backbone, whereas the LiDAR data employs a three-dimensional backbone. The BEV decoder combines these features to generate a comprehensive BEV map. This module also predicts the locations and features of navigation-related waypoints and traffic signs.

The Temporal Reasoning Module maintains a memory bank to store historical features for the purpose of managing temporal information. The method utilizes an attention-based strategy to include past features and present frame data. This approach facilitates the monitoring of objects that experience periodic occlusion and enables more precise forecasts of their future movements.

The Global Reasoning Module tracks interactions and linkages between objects and the environment in order to detect adverse events such as occlusions. In order to ensure consistency between the expected waypoints and occupancy maps, a method for modeling object-environment interaction is employed. This method includes an occupancy decoder that generates occupancy maps, as well as a consistency loss to maintain coherence. This module enhances overall perception performance and vehicle safety by studying imperceptible environments and anticipating potential dangers.

% \subsubsection{Drive in Occlusion Simulation (DOS) benchmark}
% In order to assess the performance of ReasonNet, H. Shao et al. (2023) created the Drive in Occlusion Simulation (DOS) benchmark. This benchmark is designed to recreate difficult driving scenarios where objects are concealed, presenting substantial hazards. DOS encompasses four specific occlusion situations: Parked Cars, where pedestrians may emerge from between stationary vehicles; Abrupt Deceleration, where pedestrians cause the leading vehicles to suddenly slow down; Left Turn, where a truck obstructs the visibility of oncoming traffic while making an unprotected left turn; and Red Light Infraction, where a vehicle violates a red light, remaining unseen by the ego vehicle due to trucks obstructing its view. Every scenario includes 25 distinct conditions that differ in terms of road environment and background traffic. DOS offers a thorough evaluation by considering obstructions that involve both automobiles and pedestrians. 

\subsubsection{Limitations}

Although ReasonNet greatly enhances the ability of autonomous vehicles to navigate in urban environments, it is important to investigate its inherent limits in more depth. An important issue is the framework's reliance on sensor data of excellent quality. When sensor inputs are affected by unfavorable weather circumstances like fog or heavy rain, the performance of ReasonNet may decline, emphasizing the need of having strong sensory processing and data integration capabilities.

ReasonNet has a constraint in terms of computational load due to the integration of complicated temporal and global reasoning modules. Implementing a data-intensive framework in real-time could be difficult in situations when quick decision-making is necessary. This factor could restrict the implementation of ReasonNet in less powerful hardware environments that are commonly found in commercially accessible vehicles.
